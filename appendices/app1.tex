\chapter{Open MPI extra commands}
\label{app:ompicommands}
This appendix summarizes the auxiliary commands of Open MPI not included in the
MPI standard. The list is updated with commands present in version 1.10 of Open
MPI.

The list of available commands as presented in the documentation 
\textbf{User Commands} (\emph{man page} 1) are presented in Table
\ref{tab:ompi-cmd-1}

\begin{table}[h]
\centering
\begin{tabular}{p{3cm}|p{9cm}}

\textbf{Command} & \textbf{Description} \\ \hline
\texttt{ompi-clean}
\texttt{orte-clean} & It cleans up old files and processes left by previous Open
                      MPI jobs in the local node. It kills the processes spawned
                      by a previous application and remove all temporary files.
\\ \hline 
\texttt{ompi-ps} 
\texttt{orte-ps}    & It shows the information about the active MPI jobs and
                      processes. It works universally, i.e. it displays the
                      information even if it is not the HNP.
\\ \hline
\texttt{ompi-server}
\texttt{orte-server}& It activates a server that can be contacted via the MPI
                      calls \emph{Publish\_name}/\emph{Lookup\_name}.
\\ \hline
\texttt{ompi-top}
\texttt{orte-top}   & It displays the information similar to \texttt{top} Linux
                      utility.
\\ \hline
\texttt{ompi-info}
\texttt{orte-info}  & It provides the information about the compilation and the
                      installation of Open MPI. This utility is very useful for
                      both Open MPI developers and users, in order to check
                      how Open MPI is configurated and which modules are
                      available.
\\ \hline
\texttt{opal\_wrapper} & Should not be called directly. A wrapper executable for
                        \texttt{mpicc}, \texttt{mpic++}, etc. 

\\ \hline
\texttt{orte-dvm} & It starts an \texttt{orted} for each node before submit a
                    job. The jobs may be submitted next via the 
                    \texttt{orte-submit} command. It is useful for launching
                    numerous short applications. 

\\ \hline
\texttt{orte-submit} & It submits a job into distributed daemons spawned with
                       \texttt{orte-dvm}. 

\\ \hline
\texttt{orted}       & The daemon spawned on each node.
\end{tabular}

\caption{Open MPI User commands.}
\label{tab:ompi-cmd-1}
\end{table}


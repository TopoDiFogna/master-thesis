\chapter{DistRib ILP formulation}
\label{app:distrib-ilp-formulation}
This appendix provides a formulation in \textbf{GNU MathProg} of the proposed
Integer Linear Programming problem used in the DistRib policy.

GNU MathProg is a high-level language to write mathematical models,
in particular optimization problems. It is specific to GLPK, but compatible
with the well-known \textbf{A Mathematical Programming Language (AMPL)}
\cite{fourer1990modeling}. To be precise, the GNU MathProf contains a subset
of instruction of the AMPL language.

The proposed formulation is implemented in BarbequeRTRM using the GLPK
libraries even if less intuitive than the GNU MathProg language. However, the
overhead drastically reduces, since a syntax parser of the model is not
required.

The GNU MathProg formulation is presented in Listing \ref{app2:mathprog}.

The proposed ILP has two variables, the \texttt{proc\_assigned} and the
\linebreak
\texttt{is\_assigned} variables
respectively correspond to the \(\pi\) and \(\phi\) of the mathematical
model presented in Chapter \ref{cap:integration}.

The parameters used in the formulation are:
\begin{itemize}
\item \texttt{num\_pe}: the vector containing the number of available 
processor elements per core.
\item \texttt{num\_proc}: the vector containing the number
of processes requested per application.
\item \texttt{priority}: the vector parameter
that represents \(p_a\), i.e. the priority for each application.
\item \texttt{sys\_penalty}: the vector parameter that represents \(r_s\),
i.e. an empirical number for per-system penalty.
\item \texttt{K\_dist}: the constant weight to the cost related to distribution
\item \texttt{max\_pe\_per\_system}: an arbitrarily number sufficient big to guarantee the constraints between \(\pi\) and \(\phi\) (details later).
\end{itemize}

After the definition of the cost objective function, there are four
constraints:
\begin{itemize}
\item \texttt{full\_assignment}: it assures that all the processes request of
each application is respected;
\item \texttt{resource\_availability}: it assures that the number of processes
assigned to each systems does not exceed the available processing elements;
\item \texttt{var\_association}: it bound the \(\pi\) and \(\phi\) variables in
sense that if \(\phi=1\) then  \(\pi \ge 1 \).

\item \texttt{var\_association\_rev}: the reverse bound between \(\pi\) and
\(\phi\), if \(\pi \ge 1\) then  \(\phi = 1 \).

\end{itemize}

\definecolor{britishracinggreen}{rgb}{0.0, 0.26, 0.15}
\lstdefinestyle{customc}{
  belowcaptionskip=1\baselineskip,
  breaklines=true,
  frame=L,
  xleftmargin=\parindent,
  language=C,
  showstringspaces=false,
  basicstyle=\footnotesize\ttfamily,
%  keywordstyle=\bfseries\color{green!40!black},
  commentstyle=\itshape\color{britishracinggreen},
  identifierstyle=\color{black},
  keywordstyle=\color{blue}\ttfamily,
  stringstyle=\color{orange},
alsoletter={.},
  morekeywords={%  
    set, var, in, integer, binary,param, minimize, sum, s.t.%
    }
}

\lstset{%
}%

\begin{figure}[t]
\begin{lstlisting}[style=customc]
set SYSTEMS;
set APPLICATIONS;

/* 
 * The only two necessary variables. The first one is not used in the cost
 * function but it's required for proper resource assignment.
 */
var proc_assigned{i in APPLICATIONS, j in SYSTEMS}, >=0, integer;
var is_assigned  {i in APPLICATIONS, j in SYSTEMS}, >=0, binary;

/* 
 * The various constant parameters.
 */
param num_pe     {j in SYSTEMS};
param num_proc   {i in APPLICATIONS};
param priority   {i in APPLICATIONS};
param sys_penalty{j in SYSTEMS};

param K_dist;
param max_pe_per_system; /* required for the constaints of binary variable,
                            it may be set to 1000000 or similar */

/* 
 * The minimization of the cost function.
 */
minimize cost: sum{i in APPLICATIONS} sum{j in SYSTEMS} 
                    ( priority[i] * 
                    (
                      proc_assigned[i,j] * sys_penalty[j]  + 
                      K_dist * is_assigned[i,j] 
                    ));

/* 
 * Constraints.
 */
s.t. full_assignment{i in APPLICATIONS}:
    sum{j in SYSTEMS} proc_assigned[i,j] >= num_proc[i,j];

s.t. resource_availability{j in SYSTEMS}:
    sum{i in APPLICATIONS} proc_assigned[i,j] <= num_pe[j];

s.t. var_association{i in APPLICATIONS, j in SYSTEMS}:
    is_assigned[i,j] <= proc_assigned[i,j];

s.t. var_association_rev{i in APPLICATIONS, j in SYSTEMS}:
    max_pe_per_system * is_assigned[i,j] >= proc_assigned[i,j];

\end{lstlisting}
\renewcommand\figurename{Listing}
\caption{The MathProg formulation of DistRib ILP solver.}
\label{app2:mathprog}
\end{figure}

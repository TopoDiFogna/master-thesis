% MAX 2200 CARATTERI!
\addcontentsline{toc}{chapter}{Abstract (Italian version)}
\chapter*{Abstract (Italian version)}
\markboth{Abstract (Italian version)}{}

\begin{otherlanguage}{italian}

\lettrine{I}{} sistemi High Performance Computing (HPC) sono tipicamente
caratterizzati da
un gran numero di risorse - CPU, GPU, ecc - implicando, di conseguenza, la necessità di
affrontare il problema di una loro efficace utilizzazione. In aggiunta a ciò, non
è possibile ignorare le strategie di risparmio energetico e dissipazione del calore essenziali in ambiente HPC.
Questo quadro è reso ancora più complesso dal fatto che i moderni
sistemi sono caratterizzati da un livello decrescente di affidabilità.
Per tutte queste ragioni, meccanismi e politiche poco invasive
di gestione delle risorse diventano essenziali al fine di risolvere i problemi presentati.

Proprio a causa della loro natura distribuita, i sistemi HPC utilizzano
paradigmi di programmazione
parallela, tra i quali annoveriamo uno dei più utilizzati: Message Passing
Interface (MPI). Questa tesi presenta un'estensione dell'implementazione
Open MPI, chiamata \texttt{mig}, al fine di supportare in modo trasparente
la migrazione di processi di un'applicazione.

Questo meccanismo è integrabile con un gestore delle risorse
e in questo lavoro ne viene proposto uno basato su Barbeque
Run-Time Resource Manager. Questo approccio ci consente di superare la
limitazione di MPI che impedisce la ridefinizione dell'assegnamento delle risorse computazionali a runtime una volta che l'applicazione è stata lanciata. Ciò
rappresenta anche una limitazione sia per quanto concerne l'implementazione di strategie di resilienza ai guasti, sia nei riguardi dell'uso efficace delle risorse. 

L'estensione \texttt{mig} aumenta la flessibilità introducendo una granularità
più fine di allocazione del carico di lavoro, attraverso la possibilità di
eseguire la \emph{migrazione dei processi}. A tal proposito, in letteratura esistono già molte tecniche di migrazione, ma mancano principalmente
di trasparenza
rispetto all'applicazione. In passato in Open MPI esisteva un supporto per la
tolleranza ai guasti basato su tecniche di Checkpoint/Restart, ma fu
successivamente rimosso a causa della difficile manutenibilità.

In questa tesi proponiamo il framework \texttt{mig} come un'estensione di
Open MPI per risolvere i problemi precedentemente descritti, in particolare
in termini di
trasparenza e manutenibilità. L'implementazione di una politica di allocazione
delle risorse per BarbequeRTRM è proposta come un possibile caso d'uso del framework.


\end{otherlanguage}

\addcontentsline{toc}{chapter}{Abstract}
\chapter*{Abstract}
\markboth{Abstract}{}

\lettrine{T}{he} High Performance Computing (HPC) systems typically include
a large number of computing resources - CPUs, GPUs, etc. As a consequence,
we must face with the problem of an effective utilization of them. In addition,
we cannot avoid from taking into account power saving and thermal management
strategies. The overall picture is made more complex by the fact that modern
systems are affected by decreasing level of reliability. For all these reasons,
we need effective and poorly invasive resource management mechanisms and
policies to address these issues.

Moreover, HPC systems need specific parallel programming paradigms, among these one of
the most widespread is Message Passing Interface (MPI). This thesis presents
an extension of the Open MPI implementation, called \texttt{mig} to
transparently support
the migration of application processes. This mechanism may be driven by a
resource manager and in this work an example of exploitation based on the
Barbeque Run-Time Resource Manager is proposed. This approach allows us also
to overcome a limitation of the MPI paradigm. In fact, once the application is
launched it is no more possible to redefine the assignment of computing
resource at run-time. This represents also limitation from the point of view of
effective usage of the resources and implementation of fault-tolerance
strategies.

The \texttt{mig} extension introduces more flexibility by enabling a more fine grained workload allocation, through the possibility of performing
\emph{process migration}. In this regard, a lot of  migration techniques are already available
in  literature, but they suffer from the lack of
transparency with respect to the application. In the past, Open MPI had
fault-tolerance support based on Checkpoint/Restart techniques, but they were subsequently removed due to hard maintainability requirements.

In this thesis we propose the \texttt{mig} framework as an Open MPI extension
that overcomes the aforementioned issues in terms of transparency and poor
maintainability. The implementation of a resource allocation policy for the
BarbequeRTRM is also proposed as an example of exploitation of the framework.
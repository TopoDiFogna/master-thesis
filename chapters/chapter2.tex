\chapter{State of the Art}
\label{cap:state-of-the-art}

\section{Self-Adaptive Software Systems}
\label{sec:sas}
In modern-day applications, software complexity has extremely increased thanks to the spread of highly available and faster wireless connection such as in the Internet of Things (IoT) ambit. Since software is often deployed in dynamic contexts, where requirements, environment assumptions and usage profiles varies continuously, software complexity increased over time to the point where it is often composed by a number of sub-components and/or sub-services that work together in order to offer a service to the users. This is the case of service-oriented applications -- also called Service Based Systems (SBS) -- that are composed by multiple \emph{services} and \emph{components}. In these systems, services offered by third-party providers are dynamically composed into workflows to deliver complex functionalities, so SBSs rely on self adaptation to cope with the uncertainties associated with third-party services as the loose coupling of services makes a reconfiguration feasible. Without adaptation, the application is prone to degraded performance  because of faulty components, messages lost between services or delays due to an increasing number of users.

During the past decade a lot of research has been made in this scope but the engineering of adaptive systems remains a incredible challenge.\cite{soft-eng-for-sas-2} In order to solve the problem, \textbf{Self-Adapting Software Systems (SASS)} are born. These are flexible systems that can adapt themselves to their contextual needs and can do so with the highest performance and availability. General discussion concerning the issue and the state of the art in the design and implementation have been presented.\cite{soft-eng-for-sas-2}\cite{survey-aut-comp}\cite{self-adap-soft}\cite{soft-eng-for-sas-1}\cite{soft-eng-for-sas-3}\cite{arch-based-appr-to-sas}\cite{sas-quant-ver} 

These kind of systems have some fundamental properties called auto-managing that are:
\begin{itemize}
	\item Auto-configuration
	\item Auto-recovery in case of failure
	\item Auto-optimization
	\item Auto-protection
\end{itemize}
All these properties can be grouped in two more abstract concepts which are self-awareness and context-awareness.

\textbf{Self-Awareness} is the ability of the system to be able to monitor itself in terms of available resources and behavior.

\textbf{Context-Awareness} is the ability of the system to understand the environment where it is working, using the information provided by its components, and adapt itself to all the changes that can occur during its normal operational status.
To better understand how a SASS works we need to answer some simple questions:
\begin{itemize}
	\item Who is adapting?
	\item Which adaptation is required?
	\item When is necessary to adapt?
	\item Where is needed to change something?
	\item Why is needed an adaptation?
	\item How we achieve this goal?
\end{itemize}

During the past years have been developed some dimensions that help to answer all this simple questions: \emph{Time}, \emph{Reason}, \emph{Level}, \emph{Technique} and \emph{Adaptation Control} shown in figure \ref{fig:dimensions}.
\begin{figure}[h]
	\centerline
	{\includegraphics[scale=0.50]{img/dimensions.png}}
	\caption[The Dimensions]{The Dimensions to analyze adaptation.\cite{eng-appr-sas}}
	\label{fig:dimensions}
\end{figure}

\textbf{Who is adapting?} As the name suggests, it's the system itself that changes something in order to preserve some given constraint.

\textbf{Which adaptation is required?} The \emph{Technique} dimension is the one that answers this question in fact the software engineer can change either the parameters or the system can be considered as a set of components. The former case allows to fine tuning the system at the expense of an higher complexity, the latter is called composite vision and permits the systems to cooperate exchanging algorithms and much more important, reusing components which improve performance because failed or defected components can be replaced.

\textbf{When is necessary to adapt?} The \emph{Time} dimension is crucial in this situation. There are three typical approaches: the reactive one is the more traditional one which states that an adaptation is needed only after a causative event. The other two approaches are more interesting and they are predictive and proactive. The former studies the system before any event and calculate the need of an adaptation, the latter applies and adaptation despite an event and improves the performance. From the user perspective the proactive approach is the best because it doesn't interrupt the operation of the system in any load but it is the more complicated to implement. Monitoring continuously the system is a costly task to do, on the other side an adaptive monitoring is simple that analyze only specific aspect and/or resources and intervenes only if needed.

\textbf{Where is needed to change something?} In general a SASS is composed by two main part: the the Adaptability Logic (AL) and the Managed Resource (MR). The former in general doesn't change, the latter is composed at the base of the hardware and of the software such as the operating system or, in case of distributed systems, the middleware that control the hardware; at a higher level of the application. These are the parts that require adaptation. To answer this question is needed to decide at which level the operation has to be applied without neglecting the relationship between the MR and the AL which is composed by the network that connects them and/or the view of the communication patterns. Thus \emph{Level} is the considered dimension.

\textbf{Why is needed an adaptation?} In this case, \emph{Reason} is the right dimension. There can be one or more reason because a system needs adaptation such as a change in the available resources, a change in the environment or a change in the user base of the system.

\textbf{How we achieve this goal?} The answer to this question is more complicated than the others because it needs a new topic called \emph{Adaptation Control}.

\subsection{Adaptation Control}
In the literature can be found 2 approaches: the \emph{internal approach} that intertwine the adaptation logic with the system resources, which has problems with the maintainability and scalability of the system, and the \emph{external approach} that splits the system into adaptation logic and managed resources, which increases maintainability and scalability through modularization.

The control unit needs a metric in order to decide how to adapt and in literature are present different metrics: models, rules and policies, goal or utility functions \cite{autonomic-computing}.

Another aspect of the adaptation logic is the degree of decentralization. If we have a system with limited resources then a centralized adaptation logic has to be preferred but with greater systems a decentralized AL can improve performance and  every sub-system can communicate with another with different patterns of communication. Of course hybrid technique can be made mixing the previous approaches.

\subsubsection{Adaptation Logic Issues}
As said before a SASS is composed of managed resources and the adaptation logic; it can be represented by the tuple \texttt{SASS = (AL, MR)}. \emph{AL = $a_1$, \dots, $a_n$}, with $a_i$ representing a logic element,  monitors the environment (M), analyzes the data for change (A), plans adaptation (P) and control the execution of the adaptation (E): these are known as \emph{MAPE cycle} or \emph{MAPE functionality}\cite{vision-aut-comp}. \emph{MR = $mr_1$, \dots, $mr_n$}, with $mr_i$ representing a resource, is the set of resources such as hardware with software, smart-phones, robotics or unmanned vehicles. Figure  \ref{fig:sass} shows a SASS where the dashed line represent the system border.
\begin{figure}[h]
	\centerline
	{\includegraphics[scale=0.55]{img/managed-resources.png}}
	\caption[A SASS]{A SASS (AL = Adaptation Logic, MR = Managed Resources, U = User(s), C = Context, M,A,P,E = MAPE functionality).\cite{eng-appr-sas}}
	\label{fig:sass}
\end{figure}
The dimension can therefore be mapped to the MAPE functionalities as shown in table \ref{tab:mape}.
\begin{table}[ht!b]
	\centering
	\begin{tabular}{|l||p{2.4cm}|p{2.4cm}|p{2.4cm}|p{2.4cm}|}
		\hline 
		& \textbf{Time} & \textbf{Reason} & \textbf{Level} & \textbf{Technique} \\ 
		\hline 
		\textbf{Monitoring} & Continuos & What to monitor & Identification of the levels & --- \\ 
		\hline 
		\textbf{Analyzing} & Algorithms depend on reactive or proactive dimension & Where to analyze & --- & --- \\ 
		\hline 
		\textbf{Planning} & --- & What should be influenced by planning & Adaptation plans address these levels & Plans for performing the techniques \\ 
		\hline 
		\textbf{Executing} & --- & --- & Execution of the change on the levels & Execution of the change on the levels \\ 
		\hline 
	
	\end{tabular} 
\caption[MAPE and Dimensions]{Relation of the MAPE Activities and the Dimensions}
\label{tab:mape}
\end{table}

\section{Tele Assistance: A Self-Adaptive Service-Based System}
Tele Assistance System, also known as TAS, is a research, done by the University of York \cite{teleassist}, on self-adaptation in the domain of service-based systems. Originally introduced in \cite{valid-web-serv} has already been used in the evaluation of several self-adaptation solutions \cite{valid-web-serv}, \cite{sas-quant-ver}, \cite{dyn-qos-manage}, \cite{mod-evo-conf}, \cite{conq-compl}, albeit based on ad-hoc implementations, scenarios and evaluation metrics that make the comparison of these solutions and its use to evaluate other solution very difficult.

To address these limitation the University of York implemented TAS on their Research Service Platform (ReSeP) in conjunction with concrete scenarios in order to have an immediate use in evaluation of the self-adaptation solutions.

The system provides health support to chronic condition sufferers within the comfort of their homes. TAS uses a combination of sensors embedded in a wearable device and remote services from healthcare, pharmacy and emergency service providers. As shown in Figure \ref{fig:tas-workflow}, the TAS workflow takes periodical measurements of the vital parameters of a patient and employs a third-party medical service for their analysis. The analysis result may trigger the invocation of a pharmacy service to deliver new medication to the patient or to change his/her dose of medication, or the invocation of an alarm service leading, e.g., to an ambulance being dispatched to the patient. The same alarm service can be invoked directly by the patient, by using a panic button on the wearable device.

\begin{figure}[h]
	\centerline
	{\includegraphics[scale=0.55]{img/tas-workflow.png}}
	\caption[TAS workflow]{TAS workflow}
	\label{fig:tas-workflow}
\end{figure}

They also device some generic adaptation scenarios shown in Table \ref{tab:tas-scenarios} and some metrics shown in table \ref{tab:tas-metrics}

In conclusion TAS is a reference implementation of a service based system and generic adaptation scenarios associated with different types of uncertainty. First, it aims to promote research and understanding among multiple researchers and research groups, through enabling the comparison of different self-adaptation approaches, without favouring any particular approach. Second, TAS aims to serve the advance of single research efforts by reducing the time required to evaluate self-adaptation solutions. Finally, it aims to contribute to advancing the practice of engineering self-adaptive systems, by being a realistic example of a widely used type of software system.

\begin{table}[ht!b]
	\centering
	\begin{tabular}{|l|p{3.5cm}|p{3.5cm}|p{3.5cm}|}
		\hline 
		\textbf{Scenario} & \textbf{Type of uncertainty} & \textbf{Type of adaptation} & \textbf{Type of requirements} \\ 
		\hline 
		S1 & Unpredictable environment: service failure  & Switch to equivalent service; Simultaneous invocation of several services for idempotent operation & QoS: Reliability, cost \\ 
		\hline 
		S2 & Unpredictable environment: variation of service response time & Switch to equivalent service; Simultaneous invocation of several services for idempotent operation & QoS: Performance, cost \\ 
		\hline 
		S3 & Incomplete information: new service & Use new service & QoS: Reliability, performance, cost \\ 
		\hline 
		S4 & Changing requirements: new goal & Change workflow architecture; Select new service of the change on the levels & Functional: new operation \\ 
		\hline 
		S5 & Inadequate design: wrong operation sequence & Change workflow architecture & Functional: operation sequence compliance \\ 
		\hline
		
	\end{tabular} 
	\caption[TAS Scenarios]{Generic adaptation scenarios for service-based systems}
	\label{tab:tas-scenarios}
\end{table}

\begin{table}[ht!b]
	\centering
	\begin{tabular}{|p{3cm}|p{10cm}|}
		\hline 
		\textbf{Quality Attribute} & \textbf{Metrics} \\ 
		\hline 
		Reliability & Number of failed service invocations
		Number of specific operation sequence failures
		Mean time to recovery \\ 
		\hline 
		Performance & Number of specific operation sequences exceeding allowed execution time \\ 
		\hline 
		Cost & Cumulative service invocation cost over given time period \\ 
		\hline 
		Functionalities & Number of faulty process executions \\ 
		\hline 
		
	\end{tabular} 
	\caption[TAS Metrics]{Quality attributes and metrics for the evaluation and comparison of SBS self-adaptation solutions}
	\label{tab:tas-metrics}
\end{table}

\section{The SOLAR Framework}
However working on the adaptability of a system can impact other quality attributes such as performance, reliability or maintainability and in the worst case improving adaptability can decrease part, if not all, of these attributes as stated in \cite{bass2003software}: \emph{quality attributes can never be achieved in isolation, the achievement of any one will have an effect, sometimes positive and sometimes negative, on the achievement of others}.

Find a balance between these quality attributes is often a challenging task because sometimes they're conflicting each other, e.g. lower cost and higher availability, so find an adaptability value that can meet all the requisites is, as a consequence, a challenging task too.




